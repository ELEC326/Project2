
% This LaTeX was auto-generated from MATLAB code.
% To make changes, update the MATLAB code and republish this document.

\documentclass{article}
\usepackage{graphicx}
\usepackage{color}

\sloppy
\definecolor{lightgray}{gray}{0.5}
\setlength{\parindent}{0pt}

\begin{document}

    
    
\subsection*{Contents}

\begin{itemize}
\setlength{\itemsep}{-1ex}
   \item Question 2 I
   \item Question 2 II
\end{itemize}


\subsection*{Question 2 I}

\begin{verbatim}
RV1 = load('RV1.mat'); % LOAD RV1-RV3 from file
RV1 = RV1.RV1;         % specify the variable in the MAT-file using dot notation,
                       % making it able to access each item like elements in an array
RV2 = load('RV2.mat');
RV2 = RV2.RV2;
RV3 = load('RV3.mat');
RV3 = RV3.RV3;

JPMF1 = zeros(101, 101);    % Joint pmf of RV1, RV2. JPMF is set as a Matrix as an input for surf()
JPMF2 = zeros(101, 101);    % Joint pmf of RV1, RV3

for i = 1:1000000
  JPMF1(RV1(i)+1, RV2(i)+1) = JPMF1(RV1(i)+1, RV2(i)+1)+1; %for each element both in RV1 and RV2/RV3, increase total
  JPMF2(RV1(i)+1, RV3(i)+1) = JPMF2(RV1(i)+1, RV3(i)+1)+1;
end

for i = 1:101
  for j = 1:101
    JPMF1(i, j) = JPMF1(i, j) / 1000000; % divide by number of elements to determine joint PMF
    JPMF2(i, j) = JPMF2(i, j) / 1000000;
  end
end


figure
surf(JPMF1)
title('Joint PMF of RV1, RV2')
xlabel('RV1')
ylabel('RV2')
zlabel('Joint PMF')

figure
surf(JPMF2)
title('Joint PMF of RV1, RV3')
xlabel('RV1')
ylabel('RV3')
zlabel('Joint PMF')
\end{verbatim}


\subsection*{Question 2 II}

\begin{verbatim}
%%Finding the correlation coefficient

fprintf('Correlation coefficient between RV1 and RV2:')
%%corr(RV1, RV2)
corrcoef(RV1,RV2)
fprintf('Correlation coefficient between RV1 and RV3:')
%%corr(RV1, RV3)
corrcoef(RV1,RV3)
\end{verbatim}



\end{document}
    
