
% This LaTeX was auto-generated from MATLAB code.
% To make changes, update the MATLAB code and republish this document.

\documentclass[twoside]{article}
\usepackage{setspace}
\usepackage{multicol}
\usepackage{abstract}
\usepackage{float}
\usepackage{graphicx}
\usepackage{epstopdf}
\usepackage{epsfig}
\usepackage{siunitx}
\usepackage{blindtext}
\renewcommand{\abstractnamefont}{\normalfont\bfseries} % Set the "Abstract" text to bold
\renewcommand{\abstracttextfont}{\normalfont\small\itshape} % Set the abstract itself to small italic text
\usepackage{titlesec} % Allows customization of titles
\usepackage{paralist} % Used for the compactitem environment which makes bullet points with less space between them

%\renewcommand\thesection{\Roman{section}} % Roman numerals for the sections
%\renewcommand\thesubsection{\Roman{subsection}} % Roman numerals for subsections
%\titleformat{\section}[block]{\large\scshape\centering}{\thesection.}{1em}{} % Change the look of the section titles
%\titleformat{\subsection}[block]{\large}{\thesubsection.}{1em}{} % Change the look of the section titles

%\setlength\paperheight{11.5in}			% page size
%\setlength\paperwidth{8.5in}

\setlength\textheight{9.5in}				% writeable area
\setlength\textwidth{7in}


\setlength\hoffset{0in}
\setlength\voffset{-0.75in}
\setlength\topmargin{0in}
\setlength\headheight{0in}
\setlength\headsep{0.2in}
%\setlength{\rightmargin}{0.75in}
%\setlength{\leftmargin}{0.75in}
\setlength\oddsidemargin{-0.21in}
\setlength\evensidemargin{-0.21in}
\setlength\parindent{0.0in}
\setlength\parskip{0.13in}
\setlength\marginparsep{0in}
\setlength\marginparwidth{0in}
%\setlength\footskip{0.5in}
\pagenumbering{arabic} 
\usepackage{cite}
%\newcommand{\figuredir}{figures}
%\graphicspath{{figures/}}
\singlespacing
\usepackage{longtable}
\usepackage{color}
\usepackage{wrapfig}
\usepackage{eurosym}
\usepackage{booktabs}
\usepackage{fancyhdr}

\usepackage{tikz}
\usepackage{pgfplots}
\usepackage{multirow}
\usepackage{siunitx}
\usepackage{pgfplots}
\usepackage{pgfplotstable}
\usepackage{amsmath}
\usepackage{gensymb}
\usepackage{mathpazo}

\setlength{\headheight}{0.4in}
\pagestyle{fancy}
\fancyhead{} % Blank out the default header
\fancyfoot{} % Blank out the default footer
\renewcommand{\headrulewidth}{0pt}
\cfoot{\thepage}
\usepackage{lipsum}   %usage:  \lipsum[1] will generate 1 paragraph of gibberish

\usepackage{listings}
\usepackage{color}

\definecolor{dkgreen}{rgb}{0,0.6,0}
\definecolor{gray}{rgb}{0.5,0.5,0.5}
\definecolor{mauve}{rgb}{0.58,0,0.82}

\lstset{frame=tb,
  language=Java,
  aboveskip=3mm,
  belowskip=3mm,
  showstringspaces=false,
  columns=flexible,
  basicstyle={\small\ttfamily},
  numbers=none,
  numberstyle=\tiny\color{gray},
  keywordstyle=\color{blue},
  commentstyle=\color{dkgreen},
  stringstyle=\color{mauve},
  breaklines=true,
  breakatwhitespace=true,
  tabsize=3
}

\usepackage{graphicx}
\usepackage{color}
\usepackage{listings}
\fancyhead[C]{ELEC 326$-$Probability and Random Processes,Simulation Activity 2} % Custom header text
\renewcommand{\headrulewidth}{0.4pt}% Default \headrulewidth is 0.4pt

%----------------------------------------------------------------------------------------
%	TITLE SECTION
%----------------------------------------------------------------------------------------

\title{\vspace{-5mm}\fontsize{24pt}{10pt}\selectfont\textbf{Group Simulation Activity 2}}% Article title

\author{
\large
Emily Heffernan, Kellan Rankinstein, Adel Abdel-Hamid, Eric Kailly, Alexandre Granzer-Guay  \\[4mm] % Your name
\normalsize Department of Electrical and Computer Engineering \\ % Your institution
\normalsize Queen's University at Kingston\\ 
\vspace{-1mm}}
\date{}

\sloppy
\definecolor{lightgray}{gray}{0.5}
\setlength{\parindent}{0pt}

\begin{document}
\maketitle % Insert title
\section{Question 1}
\begin{figure}[H]
\centering
	\includegraphics[width=6in,height=3in]{Q1_01.eps}
	\caption{Produced Joint PMF and Joint CDF from the provided data}
	\label{fig:Joint PMF and Joint CDF}
\end{figure}
\color{lightgray} 
\begin{lstlisting}
H =

    0.2500    0.1233
    0.4997    0.1269


x_mpdf =

    0.7498    0.2502


y_mpdf =

    0.3734    0.6266

\end{lstlisting} 
\color{black}
\begin{figure}[H]
\centering
	\includegraphics[width=4in]{Q1_02.eps}
	\caption{Joint PMF from H}
\end{figure}
Using the table provided, it can be determined that $p_x(x)=[0.75,0.25]$ for $x=[0,1]$, and that $p_y(y)=[0.375,0.625]$ for $y=[0,1]$. The estimated marginal pdfs are very close to the theoretical ones, with the probabilities differing by less than 0.01.\
\section{Question 2}
\begin{multicols}{2}
\begin{figure}[H]
\centering
	\includegraphics[width=4in]{Q2_01.eps}
\end{figure}
\begin{figure}[H]
\centering
	\includegraphics[width=4in]{Q2_02.eps}
\end{figure}
\end{multicols}
In both cases it appears the two sets of RV's are negatively correlated because small values of one RV are associated with large values of the other RV.\
\color{lightgray}
\begin{lstlisting}
Correlation coefficient between RV1 and RV2:
ans =

    1.0000   -0.7492
   -0.7492    1.0000

Correlation coefficient between RV1 and RV3:
ans =

    1.0000   -0.0005
   -0.0005    1.0000
\end{lstlisting}
\color{black}
As anticipated, the correlation coefficients are negative in both cases.
\section{Question 3}
\color{lightgray}
\begin{lstlisting}
The total PMF between rows 20 and 70 is
total =

    0.9435
\end{lstlisting}
\color{black}
    \begin{figure}[H]
\centering
	\includegraphics[width=4in]{question3_01.eps}
\end{figure}
\newpage
\section{Appendix}
\begin{multicols}{2}
\subsection{Question 1}
\begin{lstlisting}
clear all;
close all;
clc;
%Q1
%I
x=[0,1];
y=[0,1];
S=0;
for i=1:1:2
    for j=1:1:2
        if i==2
            P(j,i)=1/8;
        else if j==1
                P(j,i)=1/4;
            else
                P(j,i)=1/2;
            end
        end
    end
end


X=linspace(-2,2,21);
Y=linspace(-2,2,21);
%z=[length(X2),length(Y2)];

for X1=1:1:21
    for Y1=1:1:21
        if X1<10 || Y1<10
            z(X1,Y1)=0;
        else if X1<=16 && Y1<=16
                z(X1,Y1)=1/4;
             else if X1<=16 && Y1>16
                     z(X1,Y1)=3/4;
                 else if X1>16 && Y1<=16
                         z(X1,Y1)=3/4;
                     else
                         z(X1,Y1)=1;
                     end
                 end
            end
        end
    end
end


subplot(1,2,1)
stem3(x,y,P);
xlabel('x')
ylabel('y')
zlabel('PMF')
title('Joint PMF of X and Y')

subplot(1,2,2)
mesh(X,Y,z)
xlabel('x')
ylabel('y')
zlabel('CDF')
title('Joint CDF of X and Y')

%II

%D: Dummy variable used to aplly weight on a probability

%  !! XY array holding all X and Y  !!

for n=1:1:100000
    D(n)=randi(1000);
    if D(n)<500
        XY(1,n)=0;
        XY(2,n)=1;
    else if D(n)<750
            XY(1,n)=0;
            XY(2,n)=0;
        else if D(n)<875
                XY(1,n)=1;
                XY(2,n)=0;
            else
                XY(1,n)=1;
                XY(2,n)=1;
            end
        end
    end
end

%III

%Count occurences of each value and store 
%in vector H

H = zeros(2);

for n = 1:1:100000
    if XY(1,n)==0
        if XY(2,n)==0
            H(1,1) = H(1,1) + 1;
        else
            H(2,1) = H(2,1) + 1;
        end
    else
        if XY(2,n)==0
            H(1,2) = H(1,2) + 1;
        else
            H(2,2) = H(2,2) + 1;
        end
    end
end

H = H/100000

figure
stem3(x,y,H);
xlabel('x')
ylabel('y')
zlabel('PMF')
title('Calculated PMF of X and Y')

%When the two PDF plots are compared, it is clear that they are essentially the same, with a difference of less than 0.01 for each probability. As the number of trials approaches infinty, the estimation will approach the theoretical values.

%IV

%Calculate the marginal pdf for x and y

x_count = zeros(1,2);
y_count = zeros(1,2);

for n = 1:1:100000
    if XY(1,n) == 0
        x_count(1,1) = x_count(1,1) + 1;
    else
        x_count(1,2) = x_count(1,2) + 1;
    end
    if XY(2,n) == 0
        y_count(1,1) = y_count(1,1) + 1;
    else
        y_count(1,2) = y_count(1,2) + 1;
    end
end

x_mpdf = x_count/100000
y_mpdf = y_count/100000
\end{lstlisting}
\subsection{Question 2}
\begin{lstlisting}
RV1 = load('RV1.mat'); % LOAD RV1-RV3 from file
RV1 = RV1.RV1;         % specify the variable in the MAT-file using dot notation,
                       % making it able to access each item like elements in an array
RV2 = load('RV2.mat');
RV2 = RV2.RV2;
RV3 = load('RV3.mat');
RV3 = RV3.RV3;

JPMF1 = zeros(101, 101);    % Joint pmf of RV1, RV2. JPMF is set as a Matrix as an input for surf()
JPMF2 = zeros(101, 101);    % Joint pmf of RV1, RV3

for i = 1:1000000
  JPMF1(RV1(i)+1, RV2(i)+1) = JPMF1(RV1(i)+1, RV2(i)+1)+1; %for each element both in RV1 and RV2/RV3, increase total
  JPMF2(RV1(i)+1, RV3(i)+1) = JPMF2(RV1(i)+1, RV3(i)+1)+1;
end

for i = 1:101
  for j = 1:101
    JPMF1(i, j) = JPMF1(i, j) / 1000000; % divide by number of elements to determine joint PMF
    JPMF2(i, j) = JPMF2(i, j) / 1000000;
  end
end


figure
surf(JPMF1)
title('Joint PMF of RV1, RV2')
xlabel('RV1')
ylabel('RV2')
zlabel('Joint PMF')

figure
surf(JPMF2)
title('Joint PMF of RV1, RV3')
xlabel('RV1')
ylabel('RV3')
zlabel('Joint PMF')
%%Finding the correlation coefficient

fprintf('Correlation coefficient between RV1 and RV2:')
%%corr(RV1, RV2)
corrcoef(RV1,RV2)
fprintf('Correlation coefficient between RV1 and RV3:')
%%corr(RV1, RV3)
corrcoef(RV1,RV3)
\end{lstlisting}
\subsection{Question 3}
\begin{lstlisting}
% PART 1
load('H.mat')

total = 0;
ytotal_40 = zeros(1,101);
ytotal_10= zeros(1,101);

for i = 20:70

  for j = 1:101

    total = total +H (i,j) ;

  end
end
% printing the PMF of rows 20 to 70 and all columns within
fprintf('The total PMF between rows 20 and 70 is');
total

% PART 2

%getting CPMF for all rows of column 40
for i =1:101

    ytotal_40(i) = H(i,40);

end

%getting CPMF for all rows of column 10
for i =1:101

    ytotal_10(i) = H(i,10);

end

subplot(2,1,1);
bar(ytotal_40,'r');
axis([0,105,0,.00085]);
title('Conditional PMF for y = 40')
xlabel('Row');
ylabel('Conditional PMF');


subplot(2,1,2);
bar(ytotal_10, 'c' );
axis([0,105,0,.0000092]);
title('Conditional PMF for y = 10')
xlabel('Row');
ylabel('conditional PMF');
\end{lstlisting}
%\begin{par}
%$$e^{\pi i} + 1 = 0$$
%\end{par} \vspace{1em}
\end{multicols}
\end{document}
    
